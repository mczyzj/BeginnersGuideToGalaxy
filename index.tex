\documentclass[]{tufte-book}

% ams
\usepackage{amssymb,amsmath}

\usepackage{ifxetex,ifluatex}
\usepackage{fixltx2e} % provides \textsubscript
\ifnum 0\ifxetex 1\fi\ifluatex 1\fi=0 % if pdftex
  \usepackage[T1]{fontenc}
  \usepackage[utf8]{inputenc}
\else % if luatex or xelatex
  \makeatletter
  \@ifpackageloaded{fontspec}{}{\usepackage{fontspec}}
  \makeatother
  \defaultfontfeatures{Ligatures=TeX,Scale=MatchLowercase}
  \makeatletter
  \@ifpackageloaded{soul}{
     \renewcommand\allcapsspacing[1]{{\addfontfeature{LetterSpace=15}#1}}
     \renewcommand\smallcapsspacing[1]{{\addfontfeature{LetterSpace=10}#1}}
   }{}
  \makeatother

\fi

% graphix
\usepackage{graphicx}
\setkeys{Gin}{width=\linewidth,totalheight=\textheight,keepaspectratio}

% booktabs
\usepackage{booktabs}

% url
\usepackage{url}

% hyperref
\usepackage{hyperref}

% units.
\usepackage{units}


\setcounter{secnumdepth}{2}

% citations
\usepackage{natbib}
\bibliographystyle{apalike}

% pandoc syntax highlighting

% longtable
\usepackage{longtable,booktabs}

% multiplecol
\usepackage{multicol}

% strikeout
\usepackage[normalem]{ulem}

% morefloats
\usepackage{morefloats}


% tightlist macro required by pandoc >= 1.14
\providecommand{\tightlist}{%
  \setlength{\itemsep}{0pt}\setlength{\parskip}{0pt}}

% title / author / date
\title{A Beginners Guide to R's Galaxy}
\author{Michal J. Czyz}
\date{2018-03-15}

\usepackage{booktabs}
\usepackage{longtable}

\usepackage{amsthm}
\newtheorem{theorem}{Theorem}[chapter]
\newtheorem{lemma}{Lemma}[chapter]
\theoremstyle{definition}
\newtheorem{definition}{Definition}[chapter]
\newtheorem{corollary}{Corollary}[chapter]
\newtheorem{proposition}{Proposition}[chapter]
\theoremstyle{definition}
\newtheorem{example}{Example}[chapter]
\theoremstyle{definition}
\newtheorem{exercise}{Exercise}[chapter]
\theoremstyle{remark}
\newtheorem*{remark}{Remark}
\newtheorem*{solution}{Solution}
\begin{document}

\maketitle



{
\hypersetup{linkcolor=black}
\setcounter{tocdepth}{1}
\tableofcontents
}

\chapter{In the beginning there was only
darknes\ldots{}}\label{in-the-beginning-there-was-only-darknes}

\textbf{R} \citep{rcore2017} is one of the most common used languages in
Data Science. It is so called fourth-generation programming language
(4GL), meaning it is \emph{user-friendly}, while still quite powerful.
\textbf{R} is powered by huge open-source oriented community. Thanks to
their work, during many years of development, enormous number of
\emph{packages} (also incorrectly called \emph{libraries}) were
established, making using \textbf{R} for common works related to Data
Science easy even for Beginners.

The purpose of this document is to familiarize with \textbf{R} people
who have at least some basics in statistics or modelling and no
knowledge on programming. Thus, examples you will find in this book are
driven by making life easier for all of those who struggle with data in
their work.

To give you an example and how awesome and powerful \textbf{R} is, I
wrote whole this book in \textbf{R} using package \textbf{bookdown}
\citep{xie2016, R-bookdown}. Hoping this short description encouraged
you to dive into \emph{World of R}, we can start learning opportunities
of this programming language.

\bibliography{book.bib,packages.bib}



\end{document}
